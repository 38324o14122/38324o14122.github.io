\documentclass[twocolumn]{article}
\usepackage{ctex}
\usepackage{pgfplots}
\usepackage{placeins}
\usepackage{hyperref}
\begin{document}
\begin{table}[!ht]
    \centering
    \begin{tabular}{c|c}
        \textbf{月份} & \textbf{人数} \\ \hline
        2021年6月     & 4             \\
        2021年7月     & 1             \\
        2021年8月     & 16            \\
        2021年9月     & 0             \\
        2021年10月    & 7             \\
        2021年11月    & 5             \\
        2021年12月    & 6             \\
        2022年1月     & 0             \\
        2022年2月     & 10            \\
        2022年3月     & 0             \\
        2022年4月     & 10            \\
        2022年5月     & 0             \\
        2022年6月     & 8             \\
        2022年7月     & 5             \\
        2022年8月     & 5             \\
        2022年9月     & 5             \\
        2022年10月    & 1             \\
        2022年11月    & 10            \\
        2022年12月    & 0             \\
        2023年1月     & 5             \\
        2023年2月     & 30            \\
        2023年3月     & 5             \\
        2023年4月     & 15            \\
        2023年5月     & 3             \\
        2023年6月     & 0             \\
        2023年7月     & 5             \\
        2023年8月     & 0             \\
        2023年9月     & 5             \\
        2023年10月    & 1             \\
        2023年11月    & 12            \\
        2023年12月    & 9             \\
        2024年1月     & 6             \\
        2024年2月     & 4             \\
    \end{tabular}    
    \caption{逝世人数和月份的关系}

\end{table}
\begin{table}[!ht]
    \centering
    \begin{tabular}{c|c}
        \textbf{年份} & \textbf{人数} \\ \hline
        2021年 & 39 \\ 
        2022年 & 54 \\ 
        2023年 & 90 \\ 
        2024年 & 10 \\ 
    \end{tabular}
    \caption{逝世人数和年份的关系}
\end{table}
\begin{table}[!ht]
    \centering
    \begin{tabular}{c|c|c}
        \textbf{年份} & \textbf{平均寿命} & \textbf{寿命中位数} \\ \hline
        2021年 & 89.51 & 90 \\ 
        2022年 & 90.00 & 91 \\ 
        2023年 & 90.51 & 91 \\ 
        2024年 & 90.22 & 91 \\ 
    \end{tabular}
    \caption{平均寿命、寿命中位数和年份关系}
\end{table}

\begin{table}[!ht]
    \centering
    \begin{tabular}{c|c|c|c|c}
        \textbf{月份} & \textbf{平均寿命} & \textbf{同比} & \textbf{环比} & \textbf{} \\ \hline
        2021年6月 & 82.00 & ~ & ~ & ~ \\ 
        2021年7月 & 89.00 & 8.5\% & ~ & ~ \\ 
        2021年8月 & 91.94 & 3.3\% & ~ & ~ \\ 
        2021年9月 & ~ & ~ & ~ & ~ \\ 
        2021年10月 & 89.14 & ~ & ~ & ~ \\ 
        2021年11月 & 83.20 & -6.7\% & ~ & ~ \\ 
        2021年12月 & 93.83 & 12.8\% & ~ & ~ \\ 
        2022年1月 & ~ & ~ & ~ & ~ \\ 
        2022年2月 & 89.20 & ~ & ~ & ~ \\ 
        2022年3月 & ~ & ~ & ~ & ~ \\ 
        2022年4月 & 89.50 & ~ & ~ & ~ \\ 
        2022年5月 & ~ & ~ & ~ & ~ \\ 
        2022年6月 & 86.63 & 5.6\% & ~ & ~ \\ 
        2022年7月 & 90.80 & 2.0\% & 4.8\% & ~ \\ 
        2022年8月 & 91.40 & -0.6\% & 0.7\% & ~ \\ 
        2022年9月 & 98.80 & 8.1\% & ~ & ~ \\ 
        2022年10月 & 99.00 & 11.1\% & 0.2\% & ~ \\ 
        2022年11月 & 87.60 & 5.3\% & -11.5\% & ~ \\ 
        2022年12月 & ~ & ~ & ~ & ~ \\ 
        2023年1月 & 89.60 & ~ & ~ & ~ \\ 
        2023年2月 & 92.30 & 3.5\% & 3.0\% & ~ \\ 
        2023年3月 & 90.60 & -1.8\% & ~ & ~ \\ 
        2023年4月 & 89.13 & -0.4\% & -1.6\% & ~ \\ 
        2023年5月 & 96.33 & 8.1\% & ~ & ~ \\ 
        2023年6月 & ~ & ~ & ~ & ~ \\ 
        2023年7月 & 88.60 & -2.4\% & ~ & ~ \\ 
        2023年8月 & ~ & ~ & ~ & ~ \\ 
        2023年9月 & 90.00 & -8.9\% & ~ & ~ \\ 
        2023年10月 & 68.00 & -31.3\% & -24.4\% & ~ \\ 
        2023年11月 & 90.92 & 3.8\% & 33.7\% & ~ \\ 
        2023年12月 & 88.67 & -2.5\% & ~ & ~ \\ 
        2024年1月 & 90.17 & 0.6\% & 1.7\% & ~ \\ 
        2024年2月 & 91.25 & -1.1\% & 1.2\% & ~ \\ 
    \end{tabular}
    \caption{平均寿命和月份的关系及其同比、环比}
\end{table}

\FloatBarrier
\newpage
\FloatBarrier


\begin{figure}
    \begin{tikzpicture}
        \begin{axis}[xlabel=月份, ylabel=人数, xticklabels={., \quad, 21年6月, 8月, 10月, 12月, 22年2月, 4月, 6月, 8月, 10月, 12月, 23年2月, 4月, 6月, 8月, 10月, 12月, 24年2月}, width=18cm, x tick label style={rotate=90}]
            \addplot+[smooth] coordinates {
                (0, 4)
                (1, 1)
                (2, 16)
                (3, 0)
                (4, 7)
                (5, 5)
                (6, 6)
                (7, 0)
                (8, 10)
                (9, 0)
                (10, 10)
                (11, 0)
                (12, 8)
                (13, 5)
                (14, 5)
                (15, 5)
                (16, 1)
                (17, 10)
                (18, 0)
                (19, 5)
                (20, 30)
                (21, 5)
                (22, 15)
                (23, 3)
                (24, 0)
                (25, 5)
                (26, 0)
                (27, 5)
                (28, 1)
                (29, 12)
                (30, 9)
                (31, 6)
                (32, 4)
            };
        \end{axis}
    \end{tikzpicture}
    \caption{逝世人数和月份的关系}
\end{figure}

\FloatBarrier
\newpage
\FloatBarrier

\begin{figure}
    \begin{tikzpicture}
        \begin{axis}[xlabel=年份, ylabel=人数, width=18cm]
            \addplot+[smooth] coordinates {
                (21, 39)
                (22, 54)
                (23, 90)
                (24, 10)
            };
        \end{axis}
    \end{tikzpicture}
    \caption{逝世人数和年份的关系}
\end{figure}


\FloatBarrier
\newpage
\FloatBarrier


\begin{figure}
    \begin{tikzpicture}
        \begin{axis}[xlabel=月份, ylabel=寿命平均数, xticklabels={., \quad, 21年6月, 8月, 10月, 12月, 22年2月, 4月, 6月, 8月, 10月, 12月, 23年2月, 4月, 6月, 8月, 10月, 12月, 24年2月}, width=18cm, x tick label style={rotate=90}]
            \addplot+[smooth] coordinates {
                (0, 82)
                (1, 89)
                (2, 91.9)
                (4, 89.1)
                (5, 83.2)
                (6, 93.8)
                (8, 89.2)
                (10, 89.5)
                (12, 86.6)
                (13, 90.8)
                (14, 91.4)
                (15, 98.8)
                (16, 99)
                (17, 87.6)
                (19, 89.6)
                (20, 92.3)
                (21, 90.6)
                (22, 89.1)
                (23, 96.3)
                (25, 88.6)
                (27, 90)
                (28, 68)
                (29, 90.9)
                (30, 88.7)
                (31, 90.1)
                (32, 91.2)
            };
        \end{axis}
    \end{tikzpicture}
    \caption{寿命平均数和月份的关系}
\end{figure}

\FloatBarrier
\newpage
\FloatBarrier

\begin{figure}
    \begin{tikzpicture}
        \begin{axis}[xlabel=月份, ylabel=比值, xticklabels={., \quad, 21年6月, 8月, 10月, 12月, 22年2月, 4月, 6月, 8月, 10月, 12月, 23年2月, 4月, 6月, 8月, 10月, 12月, 24年2月}, width=18cm, x tick label style={rotate=90}]
            \addplot+[smooth, mark=*, blue] coordinates {
                (12, 0.056)
                (13, 0.020)
                (14, -0.006)
                (16, 0.111)
                (17, 0.053)
                (20, 0.035)
                (22, -0.004)
                (25, -0.024)
                (27, -0.089)
                (28, -0.313)
                (29, 0.038)
                (31, 0.006)
                (32, -0.011)
            };
            \addlegendentry{同比}
            \addplot+[smooth, mark=triangle, red] coordinates {
                (1, 0.085)
                (2, 0.033)
                (5, -0.067)
                (6, 0.128)
                (13, 0.048)
                (14, 0.007)
                (15, 0.081)
                (16, 0.002)
                (17, -0.115)
                (20, 0.030)
                (21, -0.018)
                (22, -0.016)
                (23, 0.081)
                (28, -0.244)
                (29, 0.337)
                (30, -0.025)
                (31, 0.017)
                (32, 0.012)
            };
            \addlegendentry{环比}
        \end{axis}
    \end{tikzpicture}
    \caption{寿命平均数和月份的关系的同比、环比}
\end{figure}

\FloatBarrier
\newpage

数据来自 \url{http://cpc.people.com.cn/GB/64093/87393/index1.html},截止至 2024/02/25

\end{document}